The aim of this work was to investigate coordinate structures in English in an approach not utilised in this task before -- using the Surface-syntactic Universal Dependencies annotation scheme. The choice of this approach was motivated by the goal of improving the quality of the analysed data. 

The performance of the parser trained on SUD treebanks was worse than the performance of the parser trained on UD treebanks for most of the data. The evaluation of the whole SUD-based approach (including the process of extracting coordinations from the SUD trees) showed insignificantly better results than of the UD-based approach. The results presented here, considering the Dependency Length Minimisation effect, point towards the symmetrical London style of annotating coordinations. It requires, however, taking into account that dependency lengths in the spoken language are not minimised to the same extent as in written language. It should also be considered that the dependency annotation used for syntactic analysis of the corpus influences the results obtained in the study.

Further studies could focus on possible improvements of the parsing models based on different annotation schemes and the specific effects the annotation scheme has on the analysis of coordination. As for further studies on coordinate structures, while this and previous studies on English coordinations have been fairly consistent in arguing for the symmetric approaches, specifically the London one, the argument requires further support from other languages. Researching coordinate structures in more languages can give far more insight into how language is processed. 