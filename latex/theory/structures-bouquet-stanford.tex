\subsection{Bouquet/Stanford}
The bouquet structure comes from the Stanford parser \citep{de-marneffe-etal-2006-generating}. Coordination with three conjuncts and a governor on the left would be annotated in the bouquet approach as shown in (\ref{ex:stanford}).

\begin{exe}
\centering
\ex\label{ex:stanford}
\begin{dependency}[theme = simple, baseline=-\the\dimexpr\fontdimen22\textfont2\relax]
        \begin{deptext}
        $\odot$\&$\diamond$\&$\diamond$\&$\diamond$\&,\&$\diamond$\&$\diamond$\&$\diamond$\&$\square$\&$\diamond$\&$\diamond$\&$\diamond$\\
            \end{deptext}
            \depedge{1}{2}{}
            \depedge{2}{6}{}
            \depedge{2}{10}{}
            \depedge{10}{9}{}
            \wordgroup{1}{2}{4}{c1}
            \wordgroup{1}{6}{8}{c2}
            \wordgroup{1}{10}{12}{c3}
        \end{dependency}
\end{exe}

In this approach there is a dependency connecting the governor of the coordination to the first conjunct, which is then connected to the heads of each conjunct, thus forming a bouquet. The conjunction is attached to the last conjunct in the structure. This style of annotating is one of the asymmetrical ones, since it does not treat all conjuncts of the coordination equally -- it places emphasis on the first conjunct of a coordination by making it the head of every other conjunct. 

\vspace{-0.5\treeheight}

\begin{exe}
\ex\label{ex:stanford-all}
\raisebox{-0.5ex}{
\begin{minipage}[t]{2.5\treewidth}
\begin{xlist}

\begin{multicols}{2}

\ex
\begin{minipage}[b][\treeheight]{\treewidth}
\vfill
% gov on the left, shorter conj on the left
\begin{dependency}[theme = simple, baseline=-\the\dimexpr\fontdimen22\textfont2\relax]
    \begin{deptext}
        $\odot$\&$\diamond$\&$\diamond$\&$\square$\&$\diamond$\&$\diamond$\&$\diamond$\&$\diamond$\&$\diamond$\\
    \end{deptext}
    \depedge{1}{2}{}
    \depedge{2}{5}{}
    \depedge{5}{4}{}
    \wordgroup{1}{2}{3}{}
    \wordgroup{1}{5}{9}{}
\end{dependency}
\end{minipage}

\columnbreak

\ex
\begin{minipage}[b][\treeheight]{\treewidth}
% gov on the left, shorter conj on the right
\begin{dependency}[theme = simple, baseline=-\the\dimexpr\fontdimen22\textfont2\relax]
    \begin{deptext}
        $\odot$\&$\diamond$\&$\diamond$\&$\diamond$\&$\diamond$\&$\diamond$\&$\square$\&$\diamond$\&$\diamond$\\
    \end{deptext}
    \depedge{1}{2}{}
    \depedge{2}{8}{}
    \depedge{8}{7}{}
    \wordgroup{1}{2}{6}{}
    \wordgroup{1}{8}{9}{}
\end{dependency}
\end{minipage}
\end{multicols}

\begin{multicols}{2}

\ex
\begin{minipage}[b][\treeheight]{\treewidth}
\vfill
% gov absent, shorter conj on the left
\phantom{$\odot$}\hspace{\fontdimen4\font}
\begin{dependency}[theme = simple, baseline=-\the\dimexpr\fontdimen22\textfont2\relax]
    \begin{deptext}
        $\diamond$\&$\diamond$\&$\square$\&$\diamond$\&$\diamond$\&$\diamond$\&$\diamond$\&$\diamond$\\
    \end{deptext}
    \deproot[edge height=4ex]{1}{}
    \depedge{1}{4}{}
    \depedge{4}{3}{}
    \wordgroup{1}{1}{2}{}
    \wordgroup{1}{4}{8}{}
\end{dependency}
\end{minipage}

\columnbreak

\ex
\begin{minipage}[b][\treeheight]{\treewidth}
% gov absent, shorter conj on the right
\phantom{$\odot$}\hspace{\fontdimen4\font}
\begin{dependency}[theme = simple, baseline=-\the\dimexpr\fontdimen22\textfont2\relax]
    \begin{deptext}
        $\diamond$\&$\diamond$\&$\diamond$\&$\diamond$\&$\diamond$\&$\square$\&$\diamond$\&$\diamond$\\
    \end{deptext}
    \deproot[edge height=4ex]{1}{}
    \depedge{1}{7}{}
    \depedge{7}{6}{}
    \wordgroup{1}{1}{5}{}
    \wordgroup{1}{7}{8}{}
\end{dependency}
\end{minipage}
\end{multicols}

\begin{multicols}{2}

\ex
\begin{minipage}[b][\treeheight]{\treewidth}
% gov on the right, shorter conj on the left
\phantom{$\odot$}\hspace{\fontdimen4\font}
\begin{dependency}[theme = simple, baseline=-\the\dimexpr\fontdimen22\textfont2\relax]
    \begin{deptext}
        $\diamond$\&$\diamond$\&$\square$\&$\diamond$\&$\diamond$\&$\diamond$\&$\diamond$\&$\diamond$\&$\odot$\\
    \end{deptext}
    \depedge{9}{1}{}
    \depedge{1}{4}{}
    \depedge{4}{3}{}
    \wordgroup{1}{1}{2}{}
    \wordgroup{1}{4}{8}{}
\end{dependency}
\end{minipage}

\columnbreak

\ex
\begin{minipage}[b][\treeheight]{\treewidth}
% gov on the right, shorter conj on the right
\phantom{$\odot$}\hspace{\fontdimen4\font}
\begin{dependency}[theme = simple, baseline=-\the\dimexpr\fontdimen22\textfont2\relax]
    \begin{deptext}
        $\diamond$\&$\diamond$\&$\diamond$\&$\diamond$\&$\diamond$\&$\square$\&$\diamond$\&$\diamond$\&$\odot$\\
    \end{deptext}
    \depedge{9}{1}{}
    \depedge{1}{7}{}
    \depedge{7}{6}{}
    \wordgroup{1}{1}{5}{}
    \wordgroup{1}{7}{8}{}
\end{dependency}
\end{minipage}
\end{multicols}

\end{xlist}
\end{minipage}
}
\end{exe}

\vspace{0.5\treeheight}

The diagrams in (\ref{ex:stanford-all}) show the Stanford structure with different governor positions and conjunct placements. In (\ref{ex:stanford-all}a--b) the governor is on the left with the shorter conjunct on the left in (\ref{ex:stanford-all}a) and on the right in (\ref{ex:stanford-all}b). In those diagrams, two of the dependencies drawn have the same lenght in both cases, but one of them is visibly longer in (\ref{ex:stanford-all}b). This means that, according to the DLM principle, the structure in (\ref{ex:stanford-all}a) should be preferred, so coordinations with conjuncts of visibly different lengths should have the shorter conjunct on the left. 

Within the Stanford approach this is the case for every governor position. Diagrams (\ref{ex:stanford-all}c--d) show the structure of coordination when the governor is absent, with dependencies shorter in (\ref{ex:stanford-all}c) than in (\ref{ex:stanford-all}d), so when the shorter conjunct is on the left. Diagrams (\ref{ex:stanford-all}e--f) show the structure when the governor is on the right, with dependencies shorter in (\ref{ex:stanford-all}e) than in (\ref{ex:stanford-all}f), also when the shorter conjunct is on the left. Therefore within this approach the position of the governor does not influence the preferred order of the conjuncts and the shorter conjunct should always be placed on the left. 

% to tu było ale nie wiem czy to ma sens tutaj:
% As the Universal Dependencies annotation scheme (described in Section \ref{sec:ud}) was based on the annotations produced by the Stanford Parser, the scheme now uses the Stanford style to annotate coordinations. 