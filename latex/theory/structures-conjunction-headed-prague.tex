\subsection{Conjunction-headed/Prague}
The last approach discussed here is the one associated with the Prague Dependency Treebank, called the Prague approach by \cite{popel2013coordination} or the conjunction-headed approach by \cite{prz:woz:23}.

\begin{exe}
\centering
\ex\label{ex:prague}
\begin{dependency}[theme = simple, baseline=-\the\dimexpr\fontdimen22\textfont2\relax]
            \begin{deptext}
    $\odot$\&$\diamond$\&$\diamond$\&$\diamond$\&,\&$\diamond$\&$\diamond$\&$\diamond$\&$\square$\&$\diamond$\&$\diamond$\&$\diamond$\\
            \end{deptext}
            \depedge{1}{9}{}
            \depedge{9}{2}{}
            \depedge{9}{6}{}
            \depedge{9}{10}{}
            \wordgroup{1}{2}{4}{c1}
            \wordgroup{1}{6}{8}{c2}
            \wordgroup{1}{10}{12}{c3}
        \end{dependency}
\end{exe}

This is another example of symmetrical approaches, as here again the governor treats all of the conjuncts the same way -- in this case, it does not connect to any of them. Instead, there is a dependency connecting the governor and the conjunction, which then has the conjuncts of the coordination as its dependents.

\vspace{-0.5\treeheight}
\begin{exe}
\ex\label{ex:prague-all}
\raisebox{-0.5ex}{
\begin{minipage}[t]{2.5\treewidth}
\begin{xlist}

\begin{multicols}{2}

\ex
\begin{minipage}[b][\treeheight]{\treewidth}
\vfill
% gov on the left, shorter conj on the left
\begin{dependency}[theme = simple, baseline=-\the\dimexpr\fontdimen22\textfont2\relax]
    \begin{deptext}
        $\odot$\&$\diamond$\&$\diamond$\&$\square$\&$\diamond$\&$\diamond$\&$\diamond$\&$\diamond$\&$\diamond$\\
    \end{deptext}
    \depedge{1}{4}{}
    \depedge{4}{2}{}
    \depedge{4}{5}{}
    \wordgroup{1}{2}{3}{}
    \wordgroup{1}{5}{9}{}
\end{dependency}
\end{minipage}

\columnbreak

\ex
\begin{minipage}[b][\treeheight]{\treewidth}
% gov on the left, shorter conj on the right
\begin{dependency}[theme = simple, baseline=-\the\dimexpr\fontdimen22\textfont2\relax]
    \begin{deptext}
        $\odot$\&$\diamond$\&$\diamond$\&$\diamond$\&$\diamond$\&$\diamond$\&$\square$\&$\diamond$\&$\diamond$\\
    \end{deptext}
    \depedge{1}{7}{}
    \depedge{7}{2}{}
    \depedge{7}{8}{}
    \wordgroup{1}{2}{6}{}
    \wordgroup{1}{8}{9}{}
\end{dependency}
\end{minipage}
\end{multicols}

\begin{multicols}{2}

\ex
\begin{minipage}[b][\treeheight]{\treewidth}
\vfill
% gov absent, shorter conj on the left
\phantom{$\odot$}\hspace{\fontdimen4\font}
\begin{dependency}[theme = simple, baseline=-\the\dimexpr\fontdimen22\textfont2\relax]
    \begin{deptext}
        $\diamond$\&$\diamond$\&$\square$\&$\diamond$\&$\diamond$\&$\diamond$\&$\diamond$\&$\diamond$\\
    \end{deptext}
    \deproot[edge height=4ex]{3}{}
    \depedge{3}{1}{}
    \depedge{3}{4}{}
    \wordgroup{1}{1}{2}{}
    \wordgroup{1}{4}{8}{}
\end{dependency}
\end{minipage}

\columnbreak

\ex
\begin{minipage}[b][\treeheight]{\treewidth}
% gov absent, shorter conj on the right
\phantom{$\odot$}\hspace{\fontdimen4\font}
\begin{dependency}[theme = simple, baseline=-\the\dimexpr\fontdimen22\textfont2\relax]
    \begin{deptext}
        $\diamond$\&$\diamond$\&$\diamond$\&$\diamond$\&$\diamond$\&$\square$\&$\diamond$\&$\diamond$\\
    \end{deptext}
    \deproot[edge height=4ex]{6}{}
    \depedge{6}{1}{}
    \depedge{6}{7}{}
    \wordgroup{1}{1}{5}{}
    \wordgroup{1}{7}{8}{}
\end{dependency}
\end{minipage}
\end{multicols}

\begin{multicols}{2}

\ex
\begin{minipage}[b][\treeheight]{\treewidth}
% gov on the right, shorter conj on the left
\phantom{$\odot$}\hspace{\fontdimen4\font}
\begin{dependency}[theme = simple, baseline=-\the\dimexpr\fontdimen22\textfont2\relax]
    \begin{deptext}
        $\diamond$\&$\diamond$\&$\square$\&$\diamond$\&$\diamond$\&$\diamond$\&$\diamond$\&$\diamond$\&$\odot$\\
    \end{deptext}
    \depedge{9}{3}{}
    \depedge{3}{1}{}
    \depedge{3}{4}{}
    \wordgroup{1}{1}{2}{}
    \wordgroup{1}{4}{8}{}
\end{dependency}
\end{minipage}

\columnbreak

\ex
\begin{minipage}[b][\treeheight]{\treewidth}
% gov on the right, shorter conj on the right
\phantom{$\odot$}\hspace{\fontdimen4\font}
\begin{dependency}[theme = simple, baseline=-\the\dimexpr\fontdimen22\textfont2\relax]
    \begin{deptext}
        $\diamond$\&$\diamond$\&$\diamond$\&$\diamond$\&$\diamond$\&$\square$\&$\diamond$\&$\diamond$\&$\odot$\\
    \end{deptext}
    \depedge{9}{6}{}
    \depedge{6}{1}{}
    \depedge{6}{7}{}
    \wordgroup{1}{1}{5}{}
    \wordgroup{1}{7}{8}{}
\end{dependency}
\end{minipage}
\end{multicols}

\end{xlist}
\end{minipage}
}
\end{exe}

\vspace{0.5\treeheight}

This annotation style again generates new predictions about the preferred ordering of conjuncts. When the governor is on the left, as in (\ref{ex:prague}a--b), putting the shorter conjunct on the left is preferred, the same happens in (\ref{ex:prague}c--d), when there is no governor. When the governor is on the right, however, both (\ref{ex:prague}e--f) have the same dependency lengths, therefore neither of these is preferred, because neither will shorten the dependencies.