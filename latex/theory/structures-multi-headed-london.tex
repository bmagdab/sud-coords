\subsection{Multi-headed/London}\label{sec:london}
The symmetrical approaches, as the name suggests, treat every conjunct in the coordination the same way. One of them is the multi-headed, or London approach, for which \cite{prz:woz:23} propose the name based on its appearance in Word Grammar, developed by \cite{hudson-1984, hudson-2010} at University College London.

\begin{exe}
\centering
\ex\label{ex:london}
\begin{dependency}[theme = simple, baseline=-\the\dimexpr\fontdimen22\textfont2\relax]
            \begin{deptext}
    $\odot$\&$\diamond$\&$\diamond$\&$\diamond$\&,\&$\diamond$\&$\diamond$\&$\diamond$\&$\square$\&$\diamond$\&$\diamond$\&$\diamond$\\
            \end{deptext}
            \depedge{1}{2}{}
            \depedge{1}{6}{}
            \depedge{1}{10}{}
            \depedge{10}{9}{}
            \wordgroup{1}{2}{4}{c1}
            \wordgroup{1}{6}{8}{c2}
            \wordgroup{1}{10}{12}{c3}
        \end{dependency}
\end{exe}

The symmetry of the approach comes from the fact that no conjunct is distinguished by being the only direct dependent of coordinations governor. Instead, all of the conjuncts have dependencies connecting them to the governor, and the conjunction is dependent on the last conjunct, similarly to the Stanford approach.

\vspace{-0.5\treeheight}

\begin{exe}
\ex\label{ex:london-all}
\raisebox{-0.5ex}{
\begin{minipage}[t]{2.5\treewidth}
\begin{xlist}

\begin{multicols}{2}

\ex
\begin{minipage}[b][\treeheight]{\treewidth}
\vfill
% gov on the left, shorter conj on the left
\begin{dependency}[theme = simple, baseline=-\the\dimexpr\fontdimen22\textfont2\relax]
    \begin{deptext}
        $\odot$\&$\diamond$\&$\diamond$\&$\square$\&$\diamond$\&$\diamond$\&$\diamond$\&$\diamond$\&$\diamond$\\
    \end{deptext}
    \depedge{1}{2}{}
    \depedge{1}{5}{}
    \depedge{5}{4}{}
    \wordgroup{1}{2}{3}{}
    \wordgroup{1}{5}{9}{}
\end{dependency}
\end{minipage}

\columnbreak

\ex
\begin{minipage}[b][\treeheight]{\treewidth}
% gov on the left, shorter conj on the right
\begin{dependency}[theme = simple, baseline=-\the\dimexpr\fontdimen22\textfont2\relax]
    \begin{deptext}
        $\odot$\&$\diamond$\&$\diamond$\&$\diamond$\&$\diamond$\&$\diamond$\&$\square$\&$\diamond$\&$\diamond$\\
    \end{deptext}
    \depedge{1}{2}{}
    \depedge{1}{8}{}
    \depedge{8}{7}{}
    \wordgroup{1}{2}{6}{}
    \wordgroup{1}{8}{9}{}
\end{dependency}
\end{minipage}
\end{multicols}

\begin{multicols}{2}

\ex
\begin{minipage}[b][\treeheight]{\treewidth}
\vfill
% gov absent, shorter conj on the left
\phantom{$\odot$}\hspace{\fontdimen4\font}
\begin{dependency}[theme = simple, baseline=-\the\dimexpr\fontdimen22\textfont2\relax]
    \begin{deptext}
        $\diamond$\&$\diamond$\&$\square$\&$\diamond$\&$\diamond$\&$\diamond$\&$\diamond$\&$\diamond$\\
    \end{deptext}
    \deproot[edge height=4ex]{1}{}
    \deproot[edge height=4ex]{4}{}
    \depedge{4}{3}{}
    \wordgroup{1}{1}{2}{}
    \wordgroup{1}{4}{8}{}
\end{dependency}
\end{minipage}

\columnbreak

\ex
\begin{minipage}[b][\treeheight]{\treewidth}
% gov absent, shorter conj on the right
\phantom{$\odot$}\hspace{\fontdimen4\font}
\begin{dependency}[theme = simple, baseline=-\the\dimexpr\fontdimen22\textfont2\relax]
    \begin{deptext}
        $\diamond$\&$\diamond$\&$\diamond$\&$\diamond$\&$\diamond$\&$\square$\&$\diamond$\&$\diamond$\\
    \end{deptext}
    \deproot[edge height=4ex]{1}{}
    \deproot[edge height=4ex]{7}{}
    \depedge{7}{6}{}
    \wordgroup{1}{1}{5}{}
    \wordgroup{1}{7}{8}{}
\end{dependency}
\end{minipage}
\end{multicols}

\begin{multicols}{2}

\ex
\begin{minipage}[b][\treeheight]{\treewidth}
% gov on the right, shorter conj on the left
\phantom{$\odot$}\hspace{\fontdimen4\font}
\begin{dependency}[theme = simple, baseline=-\the\dimexpr\fontdimen22\textfont2\relax]
    \begin{deptext}
        $\diamond$\&$\diamond$\&$\square$\&$\diamond$\&$\diamond$\&$\diamond$\&$\diamond$\&$\diamond$\&$\odot$\\
    \end{deptext}
    \depedge{9}{1}{}
    \depedge{9}{4}{}
    \depedge{4}{3}{}
    \wordgroup{1}{1}{2}{}
    \wordgroup{1}{4}{8}{}
\end{dependency}
\end{minipage}

\columnbreak

\ex
\begin{minipage}[b][\treeheight]{\treewidth}
% gov on the right, shorter conj on the right
\phantom{$\odot$}\hspace{\fontdimen4\font}
\begin{dependency}[theme = simple, baseline=-\the\dimexpr\fontdimen22\textfont2\relax]
    \begin{deptext}
        $\diamond$\&$\diamond$\&$\diamond$\&$\diamond$\&$\diamond$\&$\square$\&$\diamond$\&$\diamond$\&$\odot$\\
    \end{deptext}
    \depedge{9}{1}{}
    \depedge{9}{7}{}
    \depedge{7}{6}{}
    \wordgroup{1}{1}{5}{}
    \wordgroup{1}{7}{8}{}
\end{dependency}
\end{minipage}
\end{multicols}

\end{xlist}
\end{minipage}
}
\end{exe}

\vspace{0.5\treeheight}

Here, the predictions for the preferred ordering of conjuncts are different from those in asymmetrical approaches. According to this approach the placement of the governor influences the conjunct ordering, specifically the shorter conjunct will tend to be placed near the governor. In (\ref{ex:london}a) and (\ref{ex:london}b) the governor is on the left and the dependencies are shorter when the shorter conjunct is also on the left. The opposite is seen in (\ref{ex:london}e--f): the governor is on the right and putting the shorter conjunct on the right results in shorter dependencies in total. When the coordination has no governor, there is no preference for either of the options, as both have the same sum of dependency lengths.   