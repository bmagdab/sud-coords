% 1000-3000 characters including spaces
This thesis describes a corpus study on coordinate structures in English. Previous studies have shown some tendencies in how coordinations are formed and the theoretical consequences of those tendencies. The Dependency Length Minimisation effect was used to argue for the symmetric approaches to the dependency structure of coordination in one of the previous studies and another argued specifially in favour of the multi-headed approach. This current study continues the research, but using the Surface-syntactic Universal Dependencies instead of Universal Dependencies to create trees for analysis of coordinations. This choice is motivated by the strive to improve the quality of the data by using an annotation scheme that was found to possibly improve parser performance and by taking advantage of the structural qualities of the trees created according to this scheme. The chosen annotation scheme does not necessarily improve the data quality, but still allows for an analysis of coordinations. Evaluation of the discussed parsing strategies is presented, as well as the results of the coordination analysis, which provide some evidence for one of the symmetric interpretations of coordinate structures. Those results are discussed along with some unexpected findings.