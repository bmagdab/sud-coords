Coordinations are structures which consist of multiple elements, such that neither of the elements governs any of the others. Those elements are conjuncts and they can be connected with punctuation and with conjunctions such as \textsl{and}, \textsl{or}, \textsl{as well as} and others. An example of a coordination is provided in (\ref{ex:coord}), with a conjunction \textsl{and} and conjuncts: \textsl{some apples}, \textsl{the oranges your mother gave you}.

\begin{exe}
    \ex\label{ex:coord}
    Bring [[some apples] and [the oranges your mother gave you]]
\end{exe}

This work aims to investigate how coordinations should be interpreted in certain syntactic theories, specifically in dependency grammars. In dependency grammars the structure of a sentence is represented by a dependency tree, in which almost every word depends on another one -- that word is its governor. In (\ref{ex:depTree}) there is an example of a sentence with a dependency annotation.

\begin{exe}
	\ex\label{ex:depTree}
	\begin{dependency}[baseline=-\the\dimexpr\fontdimen22\textfont2\relax]
	\begin{deptext}
	The\& history\& book\& on\& the\& shelf\& is\& always\& repeating\& itself\\ 
	\end{deptext}
	\deproot{7}{root}
	\depedge[edge height=1.5cm]{7}{3}{subj}
	\depedge{7}{8}{mod}
	\depedge{7}{9}{comp:aux}
	\depedge{3}{1}{det}
	\depedge{3}{2}{compound}
	\depedge{3}{4}{mod}
	\depedge{4}{6}{comp:obj}
	\depedge{6}{5}{det}
	\depedge{9}{10}{comp:obj} %wg parsera tu powinno być udep@npmod BŁĘDNIE
	\end{dependency}
\end{exe}

\cite{prz:woz:23} have conducted a study in which they provided an argument for certain approaches to annotating coordinations in dependency grammars. To do so, they checked whether coordinate structures were formed differently depending on the position of the governor of the structure. They found that in coordinations with a governor on the left (as in (\ref{ex:GL})) it is more likely that the shorter conjunct will be also placed on the left. This means that people are more likely to form sentence (\ref{ex:GLSL}) rather than sentence (\ref{ex:GLSR}). Similarly, if the coordination has no governor (as in (\ref{ex:GN})), the shorter conjunct is more likely to be on the left, therefore there is a higher probability that sentence (\ref{ex:GNSL}) rather than (\ref{ex:GNSR}) will be uttered. 

\begin{exe}

\ex\label{ex:GL}
\begin{xlist}
\ex\label{ex:GLSL}
\textsl{Bring} [[some apples] and [the oranges your mother gave you]]

\ex\label{ex:GLSR}
\textsl{Bring} [[the oranges your mother gave you] and [some apples]]
\end{xlist}

\ex\label{ex:GN}
\begin{xlist}
\ex\label{ex:GNSL}
[[Buy some apples] or [steal as many oranges as you can hold]]

\ex\label{ex:GNSR}
[[Steal as many oranges as you can hold] or [buy some apples]]
\end{xlist}

\end{exe}

However in sentences with a governor on the right, the shorter conjunct is more likely to appear as the right one, which means that sentence (\ref{ex:GRSR}) is more likely to appear in the English language than sentence (\ref{ex:GRSL}).

\begin{exe}

\ex\label{ex:GR}
\begin{xlist}
\ex\label{ex:GRSL}
[[An apple] and [three long orange peels]] \textsl{fell} out of the bag

\ex\label{ex:GRSR}
[[Three long orange peels] and [an apple]] \textsl{fell} out of the bag
\end{xlist}

\end{exe}

Those findings were based on the Penn Treebank, which is a corpus of texts collected from the Wall Street Journal with manually added syntactic annotation. One advantage of such a resource is high quality, reliable annotation, but as a consequence the corpus is small. Adding to that the fact that it has only one specific source of texts, the corpus is not representative of the English language. 

Another paper by \cite{pbg2023} describes an attempt at replicating the results found in \cite{prz:woz:23} on a larger and more diverse corpus, namely the Corpus of Contemporary American English. Downside to using this corpus is that there is no syntactic annotation, therefore it had to be added automatically using a parser. This unfortunately lowered the quality of data. After conducting an evaluation, \cite{pbg2023} found only around half of the sampled coordinate structures to be correctly extracted from the text. 

The goal of the study described here is to tackle the replication of the aforementioned studies once again, but while trying to increase the quality of the data while still using a bigger, annotated automatically corpus. The dependency annotation in \cite{pbg2023} was made according to a scheme called Universal Dependencies. According to \cite{tuo:prz:lac:21} some automatic parsers can perform better when using another scheme called Surface-syntactic Universal Dependencies. Additionally, the structural differences between those two schemes allow for more precise heuristics for finding coordinations in parsed sentences. Therefore the difference between this study and the previous ones is the scheme in which the parser created the syntactic annotation. 

The structure of this thesis is the following: Chapter 2 explains the theoretical background behind the research questions posed here. Chapter 3 describes how the corpus data was prepared for analysis -- from training the parser to extracting the information about coordinate structures. Chapter 4 presents the analysis of the data, which is discussed in Chapter 5. Finally Chapter 6 covers the limitations of this research. 