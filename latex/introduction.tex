The aim of this work is to expand on the research on the coordinate structures in the English language. According to \cite[p. 66]{Huddleston-Pullum-2002}, ``coordination is a relation between two or more elements of syntactically equal status'', which are here called conjuncts. The joining of those conjuncts can be marked using a word called conjunction -- this could for instance be \textsl{and}, \textsl{but} or \textsl{or}. An example of a coordination is given in (\ref{ex:gov-left}) -- the word \textsl{and} serves as the conjunction, \textsl{some apples} and \textsl{the oranges your mother gave you} as the conjuncts.

\begin{exe}
    \ex\label{ex:gov-left}
    \textsl{Bring} [[some apples] and [the oranges your mother gave you]].
\end{exe}

This coordination is consistent with the observation that in English there is a tendency for the conjuncts on the left of a coordination to be shorter than the ones on the right. According to \cite{prz:woz:23}, however, the placement of the coordination's governor (the word \textsl{Bring} in the sentence (\ref{ex:gov-left})) might have an influence on this tendency: if the governor is on the left (as in (\ref{ex:gov-left})) or absent from the coordination altogether (as in (\ref{ex:no-gov})), then the tendency for the shorter conjunct to be placed on the left grows, but not if the governor is on the right (as in (\ref{ex:gov-right})).

\begin{exe}
    \ex\label{ex:no-gov}
    [[Buy some apples] or [steal as many oranges as you can hold]].
\end{exe}

\begin{exe}
    \ex\label{ex:gov-right}
    [[Three long orange peels] and [an apple]] \textsl{fell} out of the bag.
\end{exe}

Authors of the study found that this tendency changes with the length difference between the conjuncts of a coordination. For instance, the pressure to order the conjuncts a certain way was stronger in sentence (\ref{ex:gov-left}) than in the sentence \textsl{Bring apples and oranges}. The Dependency Length Minimization effect was proposed as an explanation for this. The DLM effect is a tendency observed in some languages to form sentences in a way that minimises dependency lengths, which is achieved by placing related phrases and words close to each other. 

\cite{prz:woz:23} conducted a corpus study and investigated the way in which coordinations are formed. Their observations had theoretical consequences for the possible ways of syntactic annotation of coordinations. However, the corpus they used was relatively small, therefore \cite{prz:etal:24} tried to replicate the results on a bigger corpus. Their results sharpened the conclusions from previous research, but the quality of data was low. The aim of the current study is to investigate the coordinations in English again, but using a somewhat different approach to finding the coordinations in text to improve data quality.

The structure of this thesis is the following: Chapter \ref{ch:theory} describes in more detail the theoretical aspects mentioned in the current chapter -- this includes the Dependency Length Minimization effect, different aproaches to syntactic annotation and specifically to annotation of coordinate structures. Chapter \ref{ch:technical} describes how the corpus data was prepared for analysis, i.e., how the raw text was processed to find coordinations. Chapter \ref{ch:stats} presents the statistical analysis of the data, which is discussed in Chapter \ref{ch:discussion}.